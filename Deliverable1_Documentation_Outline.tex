\documentclass{article}
\usepackage{graphicx} % Required for inserting images

\title{Deliverable 1: Documentation Outline }
\author{Group \#}
\date{Fall 2025}

\begin{document}

\maketitle
\newpage

\tableofcontents
\newpage

\section{Group Members}

Add your names, MACIDs, student numbers.
1. Amaiam Ul Haque, haquea24, 400520641
2. Jumana Ismail, ismailj, 400507846
3. Meghan Garner, garnem4, 400505701
4. Ayesha Dogar, dogara1, 400514145
5. Mehr Bagga, Baggag, 400364849
6. Diego Verdin, verdinrd, 400504430

test

\section{Part 1}

\subsection{Introduction}
Briefly describe the purpose of the pacemaker system, the goals of this deliverable, and the scope of this part.

\subsection{Requirements}
\begin{itemize}
    \item Overall system requirements (summarized from provided specification documents). It can be informal or semi-formal.
    \item Mode-specific requirements: AOO, VOO, AAI, VVI.
\end{itemize}

\subsection{Design}

In this section, you want to expand on the design decisions based on the requirements. You should be specific about your system design and how the various components relate together.

\begin{itemize}
    \item System architecture (major subsystems, hardware hiding, pin mapping).
    \item Programmable parameters (rate limits, amplitudes, pulse widths, refractory periods, etc.).
    \item Hardware inputs and outputs (signals sensed, signals controlled).
    \item State machine design for each pacing mode (with diagrams if applicable). You can also use a tabular method.
    \item Simulink diagram
    \item Screenshots of your DCM, explaining its software structure
\end{itemize}

You should also be explicit on how your design decisions map directly to the requirements.

\section{Part 2}

\subsection{Requirements Potential Changes}
Identify which requirements may evolve in the next deliverable (e.g., adding more modes, communication, new parameters).

\subsection{Design Decision Potential Changes}
List design choices that may need revisiting (e.g., choice of libraries, interface design, architecture).

\subsection{Module Description}
\begin{itemize}
    \item Purpose of the component
    \item Key functions/methods (public vs internal)
    \item Global or state variables (if any)
    \item Interactions with other components
\end{itemize}

\subsection{Testing}
Document test cases for each module. Each test case should include:

\begin{enumerate}
    \item Purpose of the test
    \item Input conditions
    \item Expected output
    \item Actual output
    \item Result (Pass/Fail)
\end{enumerate}

For instance, on the DCM side, you should test registration and login, parameter input validation, and mode selection and data storage/retrieval. This is not a complete list, depending on your system, you will need to test other components.

\subsection{GenAI Usage}
Provide a summary of any usage of GenAI in developing the model, DCM or writing this section. If you did not use GenAI tools at all, state that.

% ----------------------------------------------------
\section{General Notes}
\begin{itemize}
    \item This is a general outline based on the Deliverable 1 handout. You should make sure everything listed in the handout it is included. 
    \item Use screenshots of Simulink diagrams and DCM interface where appropriate.
    \item Ensure the requirements are traceable to design and test cases.
    \item Be concise and make things clear.
    \item You can add other sections, and you can also decide not to use this structure, however, I am including the main general sections we will expect to see.
\end{itemize}

\end{document}
